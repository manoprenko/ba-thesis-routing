Современные распределенные системы в значительной степени полагаются на наличие эффективной и отказоустойчивой сети. Алгоритмы маршрутизации широко используются в компьютерных сетях для обеспечения непрерывной и стабильной доставки пакетов при наличии аппаратных сбоев.

Наиболее распространенные link-state алгоритмы маршрутизации используют алгоритм Дейкстры для вычисления кратчайших путей и построения таблиц маршрутизации. Этот подход обеспечивает эффективную и надежную маршрутизацию, но его производительность снижается при применении к крупномасштабным сетям.

Это исследование направлено на решение проблемы производительности алгоритма Дейкстры путем небольшого ослабления требования к кратчайшим путям. В нем представлено несколько подходов, которые повышают производительность алгоритма маршрутизации, избегая при этом значительных потерь в качестве получаемых маршрутов.
